\documentclass[a4paper,10pt]{scrartcl}
\usepackage[utf8]{inputenc}
\usepackage{enumerate,graphicx}
\usepackage[ngerman]{babel}

\begin{document}

\begin{center}
  \Huge Hardwarepraktikum \\
  \large Andre Löffler, Fabian Helmschrott, Nils Wisiol \\
  \today
\end{center}


\section*{Aufgabe B}
  Unser Testsetup für diese Aufgabe war wie folgt: Zwei Maschinen wurden verwendet. Maschine A ist ein Linux-Host, Maschine B ist ein Windows-Host.
\subsection*{Was ist der Unterschied zwischen Packet- und Flow-Switching?}
  Beim Packet-Switching werden die zu \"ubertragenden Daten unabh\"angig von Struktur in Blocks verpackt. Beim Flow-Switching werden die Datenstr\"ome analysiert und je nach Bedarf die Ressourcen frei gemacht oder umverteilt.
\subsection*{Unterstützt OpenFlow das Packet-Switching?}
  OpenFlow ist in der Lage verschiedene Formen des Switching zu unterst\"utzen, darunter auch Packet- und Circuit-Switching. 
\subsection*{Bekommt h2 eine Antwort? Warum?}
  Es kommt keine Antwort, da noch keine Flows bekannt sind. Dieses Ergebnis konnten wir mit beiden Maschinen realisieren.
\subsection*{Warum bekommt h2 nun eine Antwort?}
  h2 bekommt nun eine Antwort, da die Flowregeln erstellt wurden und der Switch nun weiß, wohin gesendet werden muss. Die Flowtabelle entspricht den Erwartungen, es wurden zwei entsprechende Flows hinzugefügt -- in jede Richtung einen. Jedoch wurden für 3 Pings 7 Pakete verschickt.
\subsection*{Welche Pakete sind zu sehen? Welche Bedeutung haben sie?}
  \begin{itemize}
    \item[\textbf{[Packet In]}] Ein Paket von 2 nach 3 vom Typ OFP+ICMP geht ein.
    \item[\textbf{[Packet Out]}] Der Controller weißt den Switch an, das gerade empfangene Paket an allen Ports weiterzuversenden.
    \item[\textbf{[Paket In]}] Ein ICMP-Paket von 3 nach 2 geht ein.
    \item[\textbf{[Flow Mod]}] Der Controller weißt den Switch an, einen neue Flow anzulegen.
  \end{itemize}
  Controller und Switch begrüßen sich zunächst gegenseitig mit "Hello". Danach wird ausgehandelt, welche Features zur Verfügung stehen und welche Konfiguration verwendet wird. Anschließend wird die Verbindung mit  ``echo'' offen gehalten.
\section*{Ist ein Unterschied sichtbar?}
  \begin{tabular}{c|rr}
    & kernel space & user space \\ \hline
    Maschine A & 3660 Mb/s & 389Mb/s \\
    Maschine B & 244 Mb/s & 19,9Mb/s \\
  \end{tabular}  
\section*{Hub-Benchmark}
  Maschine A wurde auf 8 Mb/s und Maschine B auf 1 Mb/s verlangsamt. Ist das nox-Terminal nicht minimiert sondern sichtbar, so wird durch das Neuzeuchnen der Anzeige im Terminal nochmals etwa 40\% der Performance eingebüßt. 
\section*{Learn and Forward}
\begin{verbatim}
 def learn_and_forward(self, dpid, inport, packet, buf, bufid):
        """Learn MAC src port mapping, then flood or send unicast."""

        # Learn the port for the source MAC
        self.mac_to_port[mac_to_str(packet.src)] = inport
        if (mac_to_str(packet.dst) in self.mac_to_port.keys()):
            # Send unicast packet to known output port
            self.send_openflow(dpid, bufid, buf, 
                    self.mac_to_port[mac_to_str(packet.dst)], inport)
            # Later, only after learning controller works: 
            # push down flow entry and remove the send_openflow command above.
            attrs = {}
            attrs[core.IN_PORT] = inport
            attrs[core.DL_DST] = packet.dst
            action = [[openflow.OFPAT_OUTPUT, [0, 
                    self.mac_to_port[mac_to_str(packet.dst)] ]]]
            self.install_datapath_flow(dpid, attrs, 15, 90, action)
        else:
            # flood packet out everything but the input port
            self.send_openflow(dpid, bufid, buf, openflow.OFPP_FLOOD, inport)
\end{verbatim}
Maschine A erreichte mit dieser Implementierung im kernel space Modus eine Performanz von 3580 Mb/s.

\section*{Für mehrere Switches}
\begin{verbatim}
    def learn_and_forward(self, dpid, inport, packet, buf, bufid):
        """Learn MAC src port mapping, then flood or send unicast."""

        # Learn the port for the source MAC
        if (not (dpid in self.dpid_mac_to_port.keys())):
            self.dpid_mac_to_port[dpid] = {}
        self.dpid_mac_to_port[dpid][mac_to_str(packet.src)] = inport
        if (mac_to_str(packet.dst) in self.dpid_mac_to_port[dpid].keys()):
            # Send unicast packet to known output port
            self.send_openflow(dpid, bufid, buf, 
                 self.dpid_mac_to_port[dpid][mac_to_str(packet.dst)], inport)
            # Later, only after learning controller works: 
            # push down flow entry and remove the send_openflow command above.
            attrs = {}
            attrs[core.IN_PORT] = inport
            attrs[core.DL_DST] = packet.dst
            action = [[openflow.OFPAT_OUTPUT, [0, 
                     self.dpid_mac_to_port[dpid][mac_to_str(packet.dst)] ]]]
            self.install_datapath_flow(dpid, attrs, 15, 90, action)
        else:
            # flood packet out everything but the input port
            self.send_openflow(dpid, bufid, buf, openflow.OFPP_FLOOD, inport)
\end{verbatim}
  Diese Implementierung läuft auf Maschine A mit 3000 Mb/s

\section*{CBench}
  Unser eigener python-Controller lieferte bei cbench im average 4800 responses/s. Die Referenzimplementierung lieferte im average 80000 responses/s. Die nox-C++ Implementierung lieferte im  parallelen Betrieb im average 30000 responses/s, die nicht-parallele nur 2000 responses/s.
  Ein Router verteilt eingehende Pakete anhand der Ziel-IP-Adresse, während für einen Switch die MAC-Adresse entscheidend ist.
\end{document}
